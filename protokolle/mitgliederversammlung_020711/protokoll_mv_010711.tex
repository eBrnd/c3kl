\documentclass{scrartcl}
 
\usepackage[utf8]{inputenc}
\usepackage{ngerman}
 
\title{Protokoll der Mitgliederversammlung des Vereins \emph{Chaos inKL.} vom 2. Juli 2011}

\begin{document}
 
\maketitle

\section*{Tagesordnung}

\begin{enumerate}
  \setlength{\itemsep}{1pt}
  \item Begrüßung
  \item Festlegung der Tagesordnung
  \item Aussprache über Satzungsänderung \S13.4
  \item Beschluss der Satzungsänderung \S13.4
  \item Verschiedenes
\end{enumerate}

\section*{1) Begrüßung}

    Stephan Platz eröffnet als Versammlungsleiter die Versammlung und begrüßt die Anwesenden.
    Frederik Walk wird als Protokollführer bestimmt.\\
    Anwesend sind: Simon Birnbach, Mathias Dalheimer, Bernd Lietzow, Stephan Platz,\\
    Thomas S., Frederik Walk.

\section*{2) Festlegung der Tagesordnung}
    
    Stephan Platz verliest die Tagesordnung, sie wird mit 6 Ja-Stimmen, 0 Nein-Stimmen und 0 Enthaltungen angenommen.

\section*{3) Aussprache über Satzungsänderung \S13.4}

    Von Stephan Platz wird der Antrag gestellt, die Postfachadresse in Beschlussvorlage 2 auf die Adresse des Vereinsvorsitzenden des CCC Mannheim e.V., Wielandtstr. 49, 69120 Heidelberg, zu ändern.

    Danach folgt die Aussprache über die einzelnen Beschlussvorlagen. Hier werden als Argumente für den FoeBuD e.V. das deutschlandweite Agieren und die größere Öffentlichkeitsnähe angemerkt. Als Vorteile des Chaos Computer Club Mannheim e.V. werden die Stellung als Erfa-Kreis, das momentane Wachstum und die persönliche Bekanntschaft zu dessen Mitgliedern genannt.
    
\section*{4) Beschluss der Satzungsänderung \S13.4}
    
    Der Änderungsantrag zu Beschlussvorlage 2 wird mit 6 Ja-Stimmen, 0 Nein-Stimmen und 0 Enthaltungen angenommen.

    Beschlussvorlage 2 wird mit 4 zu 2 Stimmen gegenüber Beschlussvorlage 1 angenommen.

\section*{5) Verschiedenes}

    Thomas S. hat sich um mögliche Räumlichkeiten für einen Hackerspace gekümmert: Ladenlokale in der Stadt sind entweder zu teuer oder der Verein ist unerwünscht und das Pfaffwerk wird bald abgerissen. Weitere Möglichkeit ist die ``Offene Werkstatt Kaiserslautern''. Dort steht eine Werkstatt mit Werkzeug die Stundenweise und ein Seminarraum der Tageweise gemietet werden kann zur Verfügung. Als Gemeinnütziger Verein lässt sich evtl. eine dauerhaftigere Nutzung verhandeln. Vorteilhaft wäre die gesteigerte Öffentlichkeit der Veranstaltungen, entscheidender Nachteil sind die Öffnungszeiten nur zu regulären Arbeitszeiten.

    Weiterhin merkt Mathias Dalheimer an, dass es wichtig wäre in dem Raum auch Dinge dauerhaft lagern bzw. einschließen zu können und dass für einen Raum vorerst die folgenden Möglichkeiten in Betracht zu ziehen sind: Zeitweises nutzen eines fremden Raumes, dauerhaftes mieten eines kleinen Raumes, mit anderen Vereinen kooperieren oder Räume zwischenmieten.

    Als wichtiges Argument für einen Raum wird angemerkt, dass die Nähe zur Fachschaft Informatik und die Nutzung der Universitätsräume das Image eines reinen Universitätsvereins schafft und viele potenzielle Mitglieder abschreckt.

    Ein weiterer Vorschlag von Thomas S. ist, im Sommer Seminare in den Parks in Kaiserslauten zu veranstalten. Hierfür wird das anstehende ``Holzschwerter Basteln'' in Betracht gezogen.

    Letztendlich wird angemerkt, dass der Chaos Computer Club Gelder zur Verfügung hat und es herauszufinden gilt ob diese für einen Raum genutzt werden können.

    Stephan Platz schließt im folgenden die Versammlung.
\section*{}

Versammlungsleitung\hspace{5cm}Verantwortlich für das Protokoll\\
\\
\\
(Stephan Platz)\hspace{5.9cm}(Frederik Walk)
\\
\\
\end{document}
