\documentclass{scrartcl}
 
\usepackage[utf8]{inputenc}
\usepackage{ngerman}
 
\title{Protokoll der Gründungsversammlung des Vereins \emph{Chaos inKL.}}

\begin{document}
 
\maketitle

\section*{Tagesordnung}

\begin{itemize}
  \item Begrüßung
  \item Feststellung der Anzahl der stimmberechtigten Teilnehmer
  \item Genehmigung der Tagesordnung
  \item Aussprache über die Gründung des Vereins \emph{Chaos inKL.}
  \item Beratung und Verabschiedung
  \begin{itemize}
    \item einer Satzung
    \item einer Geschäftsordnung der Mitgliederversammlung
    \item einer Beitragsordnung
  \end{itemize}
  \item Wahlen des Vorstandes und Verabschiedung der Geschäftsordnung des Vorstandes
  \item Verschiedenes
\end{itemize}

\section*{1) Begrüßung}

    Mathias Dalheimer übernimmt die Rolle des Versammlungsleiters.
    Er eröffnet die Versammlung und begrüßt alle Anwesenden.

    Bitte an alle Anwesenden, sich in die Anwesenheitsliste einzutragen, und falls eine
    Mitgliedschaft in dem zu gründenden Verein gewünscht ist, Anschrift und E-Mailadresse
    anzugeben.

\section*{2) Feststellung der Anzahl der stimmberechtigten Teilnehmer}

    Es sind 10 Personen anwesend, 9 davon stimmberechtigt.

    Die Anwesenheitsliste liegt dem Protokoll als Anlage bei.

    Bernd Lietzow wird als Protokollführer bestimmt.

\section*{3) Genehmigung der Tagesordnung}


    Die Tagesordnung wird mit 9 Ja-Stimmen, 0 Nein-Stimmen und 0 Enthaltungen angenommen.

\section*{4) Aussprache über die Gründung des Vereins \emph{Chaos inKL.}}

    Sebastian Henningsen schlägt vor, dass bei der Aussprache des Namens
    \emph{Chaos inKL.} das \emph{K} und \emph{L} einzeln ausgesprochen werden.\\
    Bernd Lietzow stimmt zu.\\

    Simon Birnbach schlägt vor, bei Gesprächen mit Außenstehenden über den Verein diese
    Aussprache zu benutzen, intern ist die Aussprache egal. Die Anwesenden stimmen
    diesem Vorschlag zu.

    Kein weiterer Klärungsbedarf.

\section*{5) Beratung und Verabschiedung einer Satzung, einer Geschäftsordnung der
  Mitgliederversammlung und einer Beitragsordnung}

    Im Voraus lagen allen Anwesenden Vorschläge für Satzung, Geschäftsordnung der
    Mitgliederversammlung und Beitragsordnung vor.

    Es ergeht der Vorschlag, dass die Erhebung von Beiträgen so lange ausgesetzt wird,
    bis ein Bankkonto für den Verein eingerichtet ist. Der Vorstand wird sich zeitnah
    um die Einrichtung eines Kontos kümmern.\\
    Dieser Vorschlag wird mit 9 Ja-Stimmen, 0 Nein-Stimmen und 0 Enthaltungen angenommen.

    Es wird darüber abgestimmt, den Vorschlag der Satzung als Satzung des Vereins
    anzunehmen.\\
    Der Vorschlag wird mit 9 Ja-Stimmen, 0 Nein-Stimmen und 0
    Enthaltungen angenommen.

    Es wird darüber abgestimmt, den Vorschlag für die Geschäftsordnung als
    Geschäftsordnung des Vereins anzunehmen.\\
    Der Vorschlag wird mit 9 Ja-Stimmen, 0
    Nein-Stimmen und 0 Enthaltungen angenommen.

    Es wird darüber abgestimmt, den Vorschlag der Beitragsordnung als Beitragsordnung
    des Vereins anzunehmen.\\
    Der Vorschlag wird mit 8 Ja-Stimmen, 0 Nein-Stimmen und
    1 Enthaltung angenommen.

\section*{6) Wahlen des Vorstandes und Verabschiedung der Geschäftsordnung des Vorstandes}

    Stephan Platz wird als 1. Vorsitzender vorgeschlagen.\\
    Stephan Platz wird mit 8 Ja-Stimmen, 0 Nein-Stimmen und 1 Enthaltung zum 1.
    Vorsitzenden gewählt. Er nimmt die Wahl an.

    Simon Birnbach wird als 2. Vorsitzender vorgeschlagen.\\
    Simon Birnbach wird mit 8 Ja-Stimmen, 0 Nein-Stimmen und 1 Enthaltung zum 2.
    Vorsitzenden gewählt. Er nimmt die Wahl an.

    Maximilian Eckardt wird als Kassenwart vorgeschlagen.\\
    Maximilian Eckardt wird mit 8 Ja-Stimmen, 0 Nein-Stimmen und 1 Enthaltung zum
    Kassenwart gewählt. Er nimmt die Wahl an.

    Stephan Platz schlägt vor, den im Voraus allen Anwesenden bekannten Vorschlag für
    die Geschäftsordnung des Vorstandes als Geschäftsordnung des Vorstandes anzunehmen.\\
    Der Vorschlag wird mit 9 Ja-Stimmen, 0 Nein-Stimmen und 0 Enthaltungen angenommen.

    Das Aushändigen der Satzungskopien zur Unterschrift durch die Gründungsmitglieder
    wird nach TOP 7 verschoben.

\section*{7) Verschiedenes}

    Keine weiteren Anmerkungen.

    Ausgabe der Kopien der Satzung zur Unterschrift durch die Gründungsmitglieder.

    Mathias Dalheimer bedankt sich bei den Anwesenden und schließt die Versammlung.

\section*{}

Versammlungsleitung\\
\\
\\
\\
(Mathias Dalheimer)\\
\\
\\
Verantwortlich für das Protokoll\\
\\
\\
\\
(Bernd Lietzow)
\end{document}
