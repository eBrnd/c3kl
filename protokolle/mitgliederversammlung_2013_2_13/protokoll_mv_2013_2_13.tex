\documentclass{scrartcl}
 
\usepackage[utf8]{inputenc}
\usepackage{ngerman}
\usepackage{eurosym}
 
\title{Protokoll der Mitgliederversammlung des Vereins \emph{Chaos inKL. e.V.} vom 16. Februar 2013}

\begin{document}
 
\maketitle

\section*{Tagesordnung}

\begin{enumerate}
  \setlength{\itemsep}{1pt}
  \item Begrüßung
  \item Festlegung der Tagesordnung
  \item Bericht des Vorstandes
  \item Jahresbericht des Schatzmeisters
  \item Entlastung des Vorstandes und des Schatzmeisters
  \item Wahl eines neuen Vorstandes und Schatzmeisters
  \item Anträge
  \item Verschiedenes
\end{enumerate}

\section*{1) Begrüßung}

    \emph{Beginn der Sitzung: 15:07}\\
    \begin{itemize}
     \item Stephan Platz eröffnet als Versammlungsleiter die Versammlung und begrüßt die Anwesenden.
     \item Bernd Lietzow wird einstimmig als Protokollführer bestimmt.
     \item Die Mitgliederversammlung wurde ordnungsgemäß eingeladen, somit ist die Sitzung satzungsgemäß beschlussfähig.
     \item Anwesend sind:  Simon Birnbach, Anne Berres, Frederik Walk, Mathias Dalheimer, Jochen Kunz, Mathias Hummel, Maximilian Eckardt, Stephan Platz, Bernd Lietzow, Maurice Massar (ab TOP 6) 
    \end{itemize}

\section*{2) Festlegung der Tagesordnung}
    
    Stephan Platz verliest die Tagesordnung, sie wird mit 9 Ja-Stimmen, 0 Nein-Stimmen und 0 Enthaltungen angenommen.

\section*{3) Bericht des Vorstandes}
    In den letzten 2 Jahren hat der Vorstand folgendes geleistet:

    \begin{itemize}
    \item Eintragung des Vereins in das Vereinsregister
    \item Verein wurde als gemeinnützig anerkannt
    \item Die vorläufige Bescheinigung der Gemeinnützigkeit läuft demnächst aus und muss erneuert werden.
    \item Arduino-Workshops
    \item Anmietung des Vereinsraumes in der Ziegelstraße
    \end{itemize}

\section*{4) Jahresbericht des Schatzmeisters}

Anfang des Jahres hatten wir 3042 Euro auf dem Konto.\\
Im Moment sind es etwa 6000 Euro, darunter auch gebundene Mittel.\\
Der Raum kann nicht für das gesamte Jahr finanziert werden.\\
Es fehlen ca 702 Euro.\\
Das Geld reicht im Moment noch für sieben Monate (wir bräuchten insgesamt 19 Mitglieder, um den Raum zu tragen).

\section*{6) Entlastung des Vorstandes und des Schatzmeisters}
\emph{Maurice Massar betritt die Sitzung}\\
\\
Es wird abgestimmt, den Vorstand und den Schatzmeister zu entlasten. \\
Ergebnis: Dafür: 5, Dagegen: 0, Enthaltung: 5\\
Damit sind Vorstand und Schatzmeister entlastet.

\section*{7) Wahl eines neuen Vorstandes und Schatzmeisters}
Stephan Platz stellt sich erneut zur Wahl als neuer Vorsitzender.\\
Es wird abgestimmt, mit dem Ergebnis: Dafür: 9, Dagegen: 0, Enthaltung: 1\\
\\
Anne Berres stellt sich zur Wahl als zweite Vorsizende.\\
Es wird abgestimmt, mit dem Ergebnis: Dafür: 8, Dagegen: 0, Enthaltung: 2\\
\\
Maximilian Eckardt stellt sich zur Wahl als Schatzmeister.\\
Es wird abgestimmt, mit dem Ergebnis: Dafür: 9, Dagegen: 0, Enthaltung: 1\\
\\
Damit besteht der neue Vorstand aus Stephan Platz (1. Vorsitzender), Anne Berres (2. Vorsitzende), Maximilian Eckardt (Schatzmeister)\\
\\
Mathias Dalheimer stellt sich zur Wahl als Kassenprüfer.\\
Es wird abgestimmt, mit dem Ergebnis: Dafür: 9, Dagegen: 0, Enthaltung: 1\\
\\
Alle gewählten Personen nehmen die Wahl an.\\

\section*{8) Anträge}
\subsection*{Arduino}

Hiermit beschließt die Mitgliederversammlung vom 16.2.2013 des
Vereins ``Chaos inKL. e.V.'' die Veranstaltung eines Arduino-Workshops
und ermächtigt den Vorstand für die Durchführungen
rechtsgeschäftliche Verpflichtungen bis zu einer Höhe von 1000\officialeuro\,\,einzugehen.\\

Dafür: 9, Dagegen: 0, Enthaltung: 1

\subsection*{Getränke}

Hiermit beschließt die Mitgliederversammlung vom 16.2.2013 des
Vereins ``Chaos inKL. e.V.'' einen Getränke- und Süßigkeitenverkauf in
den vereinseigenen Räumlichkeiten einzurichten, diese mit einem
Startkapital von 200\officialeuro\,\, auszustatten und ermächtigt Frederik Walk dazu,
diese zu führen.\\

Dafür: 8, Dagegen: 0, Enthaltung: 2

\subsection*{Party}

Hiermit beschließt die Mitgliederversammlung vom 16.2.2013 des
Vereins ``Chaos inKL. e.V.'' eine Einweihungsfeier in den
vereinseigenen Räumlichkeiten durchzuführen und ermächtigt Anne
Berres diese durchzuführen und im Rahmen von 200\officialeuro\,\, nötige
Anschaffungen zu tätigen.

Es wird diskutiert:
Im Moment läuft die Planung und Organisation der Einweihungsparty nicht wie gewünscht.
Anne hat zu wenig Feedback der anderen Mitglieder erhalten, sodass der Eindruck entstand, dass das Interesse an einer Party nachgelassen hat.
Die Anwesenden kommen zu dem Konsens, dass die Party nicht wie geplant Ende Februar stattfinden kann.
Stattdessen wird ein Termin später im Jahr angestrebt.\\

Es wird über den Antrag abgestimmt, mit dem Ergebnis: 

Dafür: 4, Dagegen: 0, Enthaltung: 6

\subsection*{Løtworkshop}

Hiermit beschließt die Mitgliederversammlung vom 16.2.2013 des
Vereins ``Chaos inKL. e.V.'' einen Lötworkshop durchzuführen und
ermächtigt Anne Berres diesen durchzuführen und im Rahmen von 200\officialeuro\,\,
nötige Anschaffungen zu tätigen.\\

Es wird über den Antrag abgestimmt, mit dem Ergebnis: 

Dafür: 10, Dagegen: 0, Enthaltung: 0

\emph{Anne Berres und Mathias Dalheimer besprechen, die Planung des Lötworkshops im Anschluss an die Sitzung in der Sofaecke zu beginnen.}

\section*{9) Verschiedenes}
Mathias Dalheimer bittet um ein Meinungsbild bezüglich einer anzustrebenden Satzungsänderung.
Er schlägt vor, dass der Verein den Mitgliedern ermöglichen soll, den Mitgliedsbeitrag monatlich zu bezahlen.
Des Weiteren sollte ein Lastschriftverfahren angeboten werden.
Darüberhinaus regt er an, über die Anschaffung einer Verwaltungssoftware nachzudenken, die ein solches Verfahren erleichtert.
Beim aktuellen Modus die Beiträge zu bezahlen (Neumitglieder zahlen, wenn sie nach dem 1.7. eingetreten sind, den halben Jahresbeitrag, sonst den vollen) besteht das Problem, dass Neumitglieder den Eintritt im Verein bis zum nächsten Juli oder Januar aufschieben, damit sie nicht Mitgliedsbeiträge für Monate, in denen sie noch gar nicht Mitglied waren, bezahlen müssen.
Dieses Problem würde gelöst, indem statt eines Jahresbeitrags ein Monatsbeitrag festgelegt wird, der auch monatlich bezahlt werden kann.
Dies käme auch weniger finanzstarken Mitgliedern zu Gute, da sie nicht auf einmal einen so großen Betrag bezahlen müssten.

Es wird bemerkt, dass die Regelung des CCC Mannheim e.V. vorsieht, dass Neumitglieder den Jahresbeitrag anteilig ab dem Monat ihres Eintrags bezahlen, sonst jedoch ein Jahresbeitrag bezahlt wird.
Eine monatliche Überweisung bedeutet auch mehr Aufwand für den Schatzmeister.

Mathias Dalheimer fordert einen Rabatt für Mitglieder die jährlich zahlen.
Frederik schlägt als Prämie ein Freibier vor.
Jochen Kunz sagt, dass sich für ihn persönlich nichts ändern würde, spricht sich jedoch für eine anteilige Zahlung für Neumitglieder aus.
Frederik Walk spricht sich für eine monatliche Bezahlung mit Lastschriftverfahren aus, falls dies für den Schatzmeister keinen zu großen Mehraufwand verursacht.
Mathias Dalheimer schlägt vor, Synergien zu nutzen, da in Mannheim und in Mainz bereits Lösungen für die Verwaltung der Vereinskonten im Einsatz sind, oder aber eine Komplettlösung einzukaufen, beispielsweise von Collmex.

Das Fazit der Diskussion ist, dass wir einer entsprechenden Satzungsänderung positiv gegenüberstehen.
Mathias Dalheimer möchten den Issue-Tracker auf Github benutzen, um die Einzelheiten auszuarbeiten.

Weiterhin sieht Mathias Dalheimer ein Problem darin, dass einige Mitglieder ihre Beiträge erst nach mehrfacher Aufforderung überweisen.
Um einen Anreiz zu schaffen schlägt er vor, die Mitgliedschaft für säumige Mitglieder ruhen zu lassen, sodass diese auf einer Mitgliederversammlung kein Stimmrecht haben.

Es findet eine Diskussion darüber statt.

Das Ergebnis der Diskussion ist, dass auch diese Satzungsänderung angestrebt wird, die Einzelheiten werden ebenfalls über Github besprochen.
\\
\\
    Stephan Platz schließt im folgenden die Versammlung.\\
\\
    Ende der Sitzung: 16:07
\section*{}

Versammlungsleitung\hspace{5cm}Verantwortlich für das Protokoll\\
\\
\\
(Stephan Platz)\hspace{5.9cm}(Bernd Lietzow)
\\
\\
\end{document}
