\documentclass[a4paper, 12pt]{scrartcl}
\title{Beitragsordnung}
\subtitle{des Vereins "Chaos inKL. e.V."}
\author{}
\date{}
%% PDF SETUP
\usepackage[pdftex, bookmarks, colorlinks, breaklinks,
pdftitle=\title,pdfauthor={cd /kl},plainpages=false]{hyperref}  
\hypersetup{linkcolor=blue,citecolor=blue,filecolor=black,urlcolor=blue,plainpages=false} 
\usepackage[latin1]{inputenc}
\usepackage[T1]{fontenc}
\usepackage[ngerman]{babel}

% Use URW Garamond No. 8 as a default font. (getnonfreefonts)
\usepackage[urw-garamond]{mathdesign}
%\renewcommand{\rmdefault}{ugm}
% Optima as a sans serif font.
\renewcommand*\sfdefault{uop}
\usepackage[protrusion=true,expansion=true]{microtype}
% Recalculate page setup based on new font.
\KOMAoptions{DIV=last}
\usepackage{todonotes}
\pagestyle{plain}

\renewcommand*\thesection{\S~\arabic{section}}

\begin{document}
\maketitle

\noindent Gem�� �6 der Satzung des Vereins "`Chaos inKL. e.V."'
haben Mitglieder die von der Mitgliederversammlung
festgelegten Beitr�ge zu entrichten. Das N�here bestimmt die
nachstehende, von der Mitgliederversammlung am 10.01.2012 mit Wirkung ab
dem 10.01.2012 beschlossene Beitragsordnung.

\section{Aufnahmebeitrag}
\begin{enumerate}
	\item Der Aufnahmebeitrag wird bei Aufnahme in den Verein f�llig. Er ist bei Stellung des Aufnahmeantrages einzuzahlen. Im Falle, dass der Vorstand den Aufnahmeantrag ablehnt, wird der Aufnahmebeitrag erstattet.
	\item Der Aufnahmebeitrag betr�gt EUR 10,-.
	\item Die Mitgliedschaft beginnt, sobald der Aufnahmeantrag angenommen wurde und Aufnahmegeb�hr und Mitgliedsbeitrag bezahlt wurden.
\end{enumerate}

\section{Mitgliedsbeitrag}
\begin{enumerate}
	\item Doppelmitglieder entrichten den in \ref{dmbeitrag}.\ref{dmbeitragbetrag} festgelegten Beitrag.
	\item Die Mitglieder des Vereins, welche keine Doppelmitgliedschaft gew�hlt haben, haben einen j�hrlichen
    Mitgliedsbeitrag von EUR 120,- zu entrichten.
	\item Der Mitgliedsbeitrag kann j�hrlich vollst�ndig oder
    halbj�hrlich zur H�lfte entrichtet werden.
	\item Der Beitrag ist binnen vier Wochen nach Annahme des Aufnahmeantrages und folgend jeweils im Januar eines neuen Jahres unaufgefordert auf das Vereinskonto einzuzahlen. Bei halbj�hrlicher Zahlungsweise ist der Betrag jeweils im Januar und Juli zu entrichten
	\item Bei Aufnahme eines Mitgliedes nach dem 30.6. eines Jahres erm��igt sich der Beitrag f�r das laufende Jahr um die H�lfte des Jahresbeitrags.
	\item Im Falle nicht fristgerechter Entrichtung der Beitr�ge ruht die Mitgliedschaft. Des weiteren wird f�r jede schriftliche Mahnung eine Bearbeitungspauschale von EUR 5,- erhoben.
	\item Die Mitglieder halten ihre Mitgliedsdaten aktuell. Vorzugsweise
    werden �nderungen an die E-Mailadresse
    verwaltung@chaostreff-kaiserslautern.de mitgeteilt. 
\end{enumerate}

\section{Doppelmitgliedschaft}\label{dm}
Der Verein "`Chaos inKL. e.V."' bietet seinen Mitgliedern die Option, eine Doppelmitgliedschaft mit dem Chaos Computer Club e.V. (CCC e.V.) einzugehen. Der Verein "`Chaos inKL. e.V."' �bernimmt die Organisation der Mitgliedschaft, die sonst dem Mitglied obliegen w�rde.

\subsection{Beitr�ge}\label{dmbeitrag}
\begin{enumerate}
	\item \label{dmbeitragbetrag}F�r Mitglieder, die die Doppelmitgliedschaft w�hlen, betr�gt der Jahresbeitrag f�r den Verein "`Chaos inKL. e.V."' EUR 115,-.
	\item Mitgliedsbeitrag und Aufnahmegeb�hr des CCC e.V. richten sich nach dessen jeweils g�ltiger Beitragsordnung.
\end{enumerate}

\subsection{Organisation}
\begin{enumerate}
	\item Mitglieder, die die Doppelmitgliedschaft w�hlen, sind Mitglieder beider Vereine mit s�mtlichen Rechten und Pflichten.
	\item Doppelmitglieder bezahlen die Mitgliedsbeitr�ge f�r beide Vereine an den Verein "`Chaos inKL. e.V."'. Dieser leitet den Beitrag an den CCC e.V. weiter.
	\item Der Verein "`Chaos inKL. e.V."' �bermittelt die f�r die Mitgliederverwaltung notwendigen personenbezogenen Daten (Name, Anschrift, ggf. Nachweis f�r Erm��igung) an den CCC e.V.
	\item Endet die Mitgliedschaft im Verein "`Chaos inKL. e.V."', besteht die Mitgliedschaft im CCC e.V. unabh�ngig weiter. In diesem Fall muss die Mitgliedschaft im CCC e.V. durch das Mitglied selbstst�ndig weiterorganisiert werden.
	\item Die Mitgliedschaft im Verein "`Chaos inKL. e.V."' ist unabh�ngig von der Mitgliedschaft im CCC e.V.
\end{enumerate}

\section{Ausnahmen}
Auf Antrag kann der Vorstand in begr�ndeten Ausnahmef�llen f�r einzelne Mitglieder Ausnahmen von dieser Beitragsordnung beschlie�en.

\end{document}
