\documentclass[a4paper, 12pt]{scrartcl}

\title{Gesch�ftsordnung des Vorstands}
\subtitle{des Vereins Chaos inKL.}

\author{}
\date{}

%% PDF SETUP
\usepackage[pdftex, bookmarks, colorlinks, breaklinks,
pdftitle=\title,pdfauthor={C3KL},plainpages=false]{hyperref}  
\hypersetup{linkcolor=blue,citecolor=blue,filecolor=black,urlcolor=blue,plainpages=false} 
\usepackage[latin1]{inputenc}
\usepackage[T1]{fontenc}
\usepackage[ngerman]{babel}

% Use URW Garamond No. 8 as a default font. (getnonfreefonts)
\usepackage[urw-garamond]{mathdesign}
%\renewcommand{\rmdefault}{ugm}
% Optima as a sans serif font.
\renewcommand*\sfdefault{uop}
\usepackage[protrusion=true,expansion=true]{microtype}
% Recalculate page setup based on new font.
\KOMAoptions{DIV=last}
\usepackage{todonotes}
\pagestyle{plain}

\renewcommand*\thesection{\S~\arabic{section}}

\begin{document}
\maketitle
In Ausf�llung und Erg�nzung des von der Satzung des Vereins "`Chaos
inKL."' vorgegebenen Rahmens wird folgende Gesch�ftsordnung erlassen:

\section{Versammlungsordnung}
\begin{enumerate}
	\item Der Vorstand soll einmal im Quartal tagen.
\item Der Vorstand w�hlt aus seinen Reihen einen Protokollf�hrer, der den
Ablauf der Vorstandssitzung protokolliert.
	\item �ber den Verlauf der Vorstandssitzungen ist eine Niederschrift
anzufertigen, die von allen Anwesenden zu unterzeichnen ist. Die
Niederschrift ist innerhalb einer Woche den Mitgliedern schriftlich oder
per E-Mail zur Verf�gung zu stellen. Erfolgt nach der Ver�ffentlichung der
Niederschrift innerhalb von vier Wochen kein Einspruch, gilt diese als
genehmigt.
\end{enumerate}

\section{Zust�ndigkeiten des Schatzmeisters}
\begin{enumerate}
	\item Der Schatzmeister �berwacht die Haushaltsf�hrung und verwaltet das
Verm�gen des Vereins. Er hat auf eine sparsame und wirtschaftliche
Haushaltsf�hrung hinzuwirken.
	\item Mit dem Ablauf des Gesch�ftsjahres stellt er unverz�glich die Abrechnung
sowie die Verm�gens�bersicht und sonstige Unterlagen von
wirtschaftlichem Belang dem Rechnungspr�fer des Vereins zur Pr�fung zur
Verf�gung.
\end{enumerate}

\section{Beir�te}
Der Vorstand kann "`Fachliche Beir�te"' oder "`Wissenschaftliche Beir�te"' einrichten,
die f�r den Verein beratend und unterst�tzend t�tig werden; in die Beir�te k�nnen
auch Nicht-Mitglieder berufen werden.

\section{Inkrafttreten}
Diese Gesch�ftsordnung wurde durch die 
Gr�ndungsversammlung 
%Mitgliederversammlung 
vom 26.02.2011 beschlossen und tritt mit sofortiger Wirkung in Kraft.\\
%Diese Gesch�ftsordnung ersetzt alle vorher beschlossenen Gesch�ftsordnungen.
\end{document}
